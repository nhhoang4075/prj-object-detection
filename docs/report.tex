\documentclass[12pt,a4paper]{article}
\usepackage[utf8]{inputenc}
\usepackage[vietnamese]{babel}
\usepackage{graphicx}
\usepackage[hidelinks]{hyperref}
\usepackage{listings}
\usepackage{xcolor}
\usepackage{geometry}
\usepackage{fancyhdr}
\usepackage{titlesec}
\usepackage{booktabs}
\usepackage{float}
\usepackage{amsmath}
\usepackage{mathtools}

\geometry{margin=2.5cm, headheight=15pt}
\graphicspath{{images/}}

\definecolor{codegreen}{rgb}{0,0.6,0}
\definecolor{codegray}{rgb}{0.5,0.5,0.5}
\definecolor{codepurple}{rgb}{0.58,0,0.82}
\definecolor{backcolour}{rgb}{0.95,0.95,0.95}

\lstdefinestyle{mystyle}{
    backgroundcolor=\color{backcolour},
    commentstyle=\color{codegreen},
    keywordstyle=\color{blue},
    numberstyle=\tiny\color{codegray},
    stringstyle=\color{codepurple},
    basicstyle=\ttfamily\small,
    breakatwhitespace=false,
    breaklines=true,
    captionpos=b,
    keepspaces=true,
    numbers=left,
    numbersep=5pt,
    showspaces=false,
    showstringspaces=false,
    showtabs=false,
    tabsize=2,
    frame=single
}
\lstset{style=mystyle}

\pagestyle{fancy}
\fancyhf{}
\rhead{Project 1}
\lhead{Nguyễn Huy Hoàng}
\rfoot{Trang \thepage}

\begin{document}

\begin{titlepage}
    \centering
    \vspace*{0.5cm}

    \textbf{\Large ĐẠI HỌC BÁCH KHOA HÀ NỘI}\\[0.3cm]
    \textbf{\large TRƯỜNG CÔNG NGHỆ THÔNG TIN VÀ TRUYỀN THÔNG}\\[1.5cm]

    \includegraphics[width=0.25\textwidth]{hust_logo.jpg}\\[0.5cm]

    \rule{\linewidth}{0.5mm}\\[0.5cm]
    {\LARGE \textbf{BÁO CÁO PROJECT 1}}\\[0.3cm]
    {\Large \textbf{XÂY DỰNG MÔ HÌNH AI PHÁT HIỆN VẬT THỂ NGUY HIỂM}}\\[0.3cm]
    \rule{\linewidth}{0.5mm}\\[2cm]

    \begin{tabular}{ll}
        \textbf{Sinh viên thực hiện:} & Nguyễn Huy Hoàng \\[0.3cm]
        \textbf{Mã số sinh viên:} & 20235336 \\[0.3cm]
        \textbf{Mã lớp học:} & 755566 \\[0.3cm]
        \textbf{Giáo viên hướng dẫn:} & Hoàng Việt Dũng \\
    \end{tabular}\\[3cm]

    \textbf{Hà Nội, 2025}

\end{titlepage}

\tableofcontents
\newpage

\section{Tổng quan dự án}

\subsection{Mục tiêu}
Xây dựng hệ thống camera an ninh phát hiện và cảnh báo vật thể nguy hiểm (dao, kéo) trong thời gian thực, với giao diện web cho phép theo dõi và quản lý các cảnh báo.

\subsection{Công nghệ sử dụng}
\begin{itemize}
    \item \textbf{YOLOv11}: Model nhận diện vật thể real-time
    \item \textbf{OpenCV}: Xử lý ảnh và video
    \item \textbf{FastAPI}: Backend web application
    \item \textbf{WebSocket}: Truyền dữ liệu real-time giữa client và server
    \item \textbf{Python}: Ngôn ngữ lập trình chính
\end{itemize}

\section{Model YOLOv11}

\subsection{Giới thiệu}
YOLOv11 là phiên bản mới nhất trong dòng YOLO của Ultralytics, được phát hành vào năm 2024. Model này mang đến những cải tiến đáng kể về kiến trúc và phương pháp training, tối ưu cho nhiều tác vụ computer vision khác nhau.

\begin{figure}[H]
    \centering
    \includegraphics[width=0.9\textwidth]{yolo_revolution.png}
    \caption{Sự phát triển của các phiên bản YOLO qua các năm}
    \label{fig:yolo_revolution}
\end{figure}

\subsection{Kiến trúc}
YOLOv11 có kiến trúc 3 phần chính: Backbone, Neck và Head.

\begin{figure}[H]
    \centering
    \includegraphics[width=0.95\textwidth]{yolo_architecture.png}
    \caption{Kiến trúc tổng quan của YOLOv11}
    \label{fig:yolo_architecture}
\end{figure}

\subsubsection{Backbone}
Backbone sử dụng DarkNet và DarkFPN để trích xuất features. Các thành phần chính:
\begin{itemize}
    \item \textbf{Conv Block}: Gồm Convolutional layer + Batch Normalization + SiLU activation
    \item \textbf{Bottleneck}: Hai Conv blocks nối tiếp với residual connection (tương tự ResNet)
    \item \textbf{C3K2 Block}: Phiên bản cải tiến của C2F (YOLOv8), sử dụng kernel 3x3 nhỏ hơn để trích xuất features hiệu quả hơn
    \item \textbf{SPPF (Spatial Pyramid Pooling Fast)}: Pooling features ở nhiều scales khác nhau bằng multiple max-pooling operations
\end{itemize}

\begin{figure}[H]
    \centering
    \includegraphics[width=0.5\textwidth]{silu.png}
    \caption{Hàm kích hoạt SiLU (Sigmoid Linear Unit) - $f(x) = x \cdot \sigma(x)$}
    \label{fig:silu}
\end{figure}

Backbone trích xuất 3 mức features:
\begin{itemize}
    \item P3 ($80\times80$): high-level features
    \item P4 ($40\times40$): medium-level features
    \item P5 ($20\times20$): low-level features
\end{itemize}

\subsubsection{Neck}
Neck xử lý features từ backbone trước khi đưa vào head. Thành phần quan trọng:
\begin{itemize}
    \item \textbf{C2PSA (Cross Stage Partial with Spatial Attention)}: Block mới trong YOLOv11, kết hợp Attention mechanism + Conv + FFN để tập trung vào các vùng quan trọng trong feature map
    \item \textbf{Upsampling \& Concatenation}: Upsample P5 $\rightarrow$ concat P4 $\rightarrow$ upsample $\rightarrow$ concat P3
\end{itemize}

C2PSA giúp model focus vào spatial information, cải thiện khả năng phát hiện small objects và partially occluded objects.

\subsubsection{Head}
Head nhận 3 feature maps và thực hiện prediction:
\begin{itemize}
    \item \textbf{Detection Head}: Gồm DFL (Distribution Focal Loss), Box Detection, Class Detection
    \item \textbf{Multi-scale Predictions}: Predictions từ P3, P4, P5 giúp phát hiện objects ở nhiều kích thước
    \item \textbf{Anchor-based Detection}: Sử dụng anchor points và strides để xác định vị trí bounding boxes
\end{itemize}

\subsection{Các phiên bản}

\begin{table}[H]
\centering
\begin{tabular}{lcccc}
\toprule
\textbf{Phiên bản} & \textbf{Params} & \textbf{GFLOPs} & \textbf{mAP} & \textbf{Ứng dụng} \\
\midrule
YOLOv11n (Nano) & 2.6M & 6.5 & 39.5 & Edge devices \\
YOLOv11s (Small) & 9.4M & 21.5 & 47.0 & Balanced \\
YOLOv11m (Medium) & 20.1M & 68.0 & 51.5 & General purpose \\
YOLOv11l (Large) & 25.3M & 86.9 & 53.4 & High accuracy \\
YOLOv11x (Extra Large) & 56.9M & 194.9 & 54.7 & Maximum accuracy \\
\bottomrule
\end{tabular}
\caption{Các phiên bản YOLOv11}
\end{table}

\subsection{Fine-tuning}

Quy trình fine-tune YOLOv11 với custom dataset:

\textbf{Bước 1: Chuẩn bị Dataset}
\begin{itemize}
    \item Thu thập images chứa các vật thể nguy hiểm (dao, kéo)
    \item Annotate theo format của YOLO
    \item Chia dataset: train/val/test (75/15/10)
\end{itemize}

\textbf{Bước 2: Cấu hình training}
\begin{itemize}
    \item Tạo file data.yaml định nghĩa paths và classes
    \item Chọn pretrained weights: yolov11n.pt
\end{itemize}

\textbf{Bước 3: Training}
\begin{lstlisting}[language=Python]
from ultralytics import YOLO

model = YOLO("yolo11s.pt")
model.train(
    data="./dataset/merge.yolov11/data.yaml",
    epochs=50,
    patience=10,
    batch=-1,
    imgsz=640,
    project=output_dir,
    name="train",
    
    # === Loss ===
    box=7.5,
    cls=1.5,      # 0.5 -> 1.5
    dfl=1.5,
    
    # === Regularization ===
    dropout=0.2,
    weight_decay=0.0005,
    
    # === Augmentation ===
    # Geometric
    fliplr=0.5,
    degrees=15,       # Rotate +/-15 deg
    scale=0.5,        # Zoom 50-150%
    translate=0.1,
    
    # Mosaic & Mixup
    mosaic=1.0,
    mixup=0.15,
    copy_paste=0.1,
    
    # Color
    hsv_h=0.015,
    hsv_s=0.7,
    hsv_v=0.4,
    
    # Random erasing
    erasing=0.3,
)
\end{lstlisting}

\textbf{Bước 4: Evaluation}
\begin{itemize}
    \item Đánh giá trên test set: mAP, precision, recall
    \item Export model cho deployment
\end{itemize}

\section{Xây dựng Web Application}

\subsection{Kiến trúc hệ thống}

\begin{center}
\fbox{
\begin{minipage}{0.35\textwidth}
\textbf{Browser}
\begin{itemize}
    \item getUserMedia()
    \item Canvas capture
    \item Draw boxes
\end{itemize}
\end{minipage}
}
$\xleftrightarrow{\text{WebSocket}}$
\fbox{
\begin{minipage}{0.35\textwidth}
\textbf{FastAPI Server}
\begin{itemize}
    \item YOLO Model
    \item Detection
    \item Auto capture
\end{itemize}
\end{minipage}
}
\end{center}

\subsection{Backend}

Backend được xây dựng với FastAPI, cung cấp các chức năng:

\begin{itemize}
    \item \textbf{WebSocket endpoint}: Nhận frames từ browser, chạy detection, trả về kết quả
    \item \textbf{Auto capture}: Tự động lưu ảnh khi phát hiện vật thể nguy hiểm
    \item \textbf{REST API}: Quản lý các ảnh đã capture (list, delete)
\end{itemize}

\subsection{Frontend}

Frontend là một Web Application với hai trang chính:

\textbf{1. Trang Detection}
\begin{itemize}
    \item Hiển thị video từ webcam real-time
    \item Vẽ bounding boxes lên các vật thể được phát hiện
    \item Màu đỏ cho vật thể nguy hiểm (dao, kéo)
    \item Hiển thị confidence score
\end{itemize}

\begin{figure}[H]
    \centering
    \includegraphics[width=0.9\textwidth]{detection_page.png}
    \caption{Trang Detection - Phát hiện vật thể nguy hiểm real-time}
    \label{fig:detection_page}
\end{figure}

\textbf{2. Trang Captures}
\begin{itemize}
    \item Gallery hiển thị các ảnh đã capture
    \item Xem ảnh full-screen với modal
    \item Xóa từng ảnh hoặc xóa tất cả
\end{itemize}

\begin{figure}[H]
    \centering
    \includegraphics[width=0.9\textwidth]{capture_page.png}
    \caption{Trang Captures - Quản lý các ảnh đã capture}
    \label{fig:capture_page}
\end{figure}

\subsection{Tính năng Auto Capture}

Hệ thống tự động chụp và lưu ảnh khi phát hiện vật thể nguy hiểm:

\begin{itemize}
    \item Chỉ capture khi phát hiện dao hoặc kéo
    \item Cooldown 3 giây giữa các lần capture để tránh spam
    \item Ảnh được lưu với bounding boxes và timestamp
    \item Tên file: \texttt{YYYYMMDD\_HHMMSS\_classes.jpg}
\end{itemize}

\section{Cấu trúc dự án}

\begin{lstlisting}[basicstyle=\ttfamily\small, frame=none, numbers=none]
Project/
|-- web/                       # Web application
|   |-- app.py                 # FastAPI backend
|   +-- static/                # Frontend files
|-- docs/                      # Documentation
|   |-- report.tex             # LaTeX report
|   +-- images/                # Report images
|-- captures/                  # Auto-captured images
|-- dataset/                   # Training datasets
|-- results/                   # Training results
|-- main.py                    # Standalone detection
|-- preprocessing.py           # Dataset preparation
|-- yolov11-finetune.ipynb     # Finetuning notebook
|-- requirements.txt           # Dependencies
+-- README.md                  # Documentation
\end{lstlisting}

\section{Hướng dẫn}

\subsection{Cài đặt}

\begin{lstlisting}[language=bash]
# Create virtual environment
python -m venv .venv
source .venv/bin/activate

# Install dependencies
pip install -r requirements.txt
\end{lstlisting}

\subsection{Chạy Web Application}

\begin{lstlisting}[language=bash]
cd web
uvicorn app:app --reload
\end{lstlisting}

Mở trình duyệt tại \url{http://localhost:8000}

\subsection{Training Model}

\begin{lstlisting}[language=bash]
# Prepare dataset
python preprocessing.py

# Fine-tune model (Jupyter notebook)
jupyter notebook yolov11-finetune.ipynb
\end{lstlisting}

\section{Kết quả}

\subsection{Model Performance}

\begin{figure}[H]
    \centering
    \begin{minipage}{0.48\textwidth}
        \centering
        \includegraphics[width=\textwidth]{pretrain_inference.jpg}
        \caption{Pretrained model (chưa fine-tune)}
        \label{fig:pretrain}
    \end{minipage}
    \hfill
    \begin{minipage}{0.48\textwidth}
        \centering
        \includegraphics[width=\textwidth]{finetuned_inference.jpg}
        \caption{Finetuned model (đã fine-tune)}
        \label{fig:finetuned}
    \end{minipage}
\end{figure}

Model sau khi fine-tune có khả năng nhận diện chính xác dao và kéo, trong khi model pretrained chỉ nhận diện được các object chung từ COCO dataset và có độ chính xác thấp.

\begin{figure}[H]
    \centering
    \includegraphics[width=0.7\textwidth]{confusion_matrix_normalized.png}
    \caption{Confusion Matrix (Normalized) - Đánh giá hiệu suất phân loại của model sau khi fine-tune}
    \label{fig:confusion_matrix}
\end{figure}

\subsection{Web Application}

\begin{itemize}
    \item Real-time detection qua WebSocket
    \item Auto capture khi phát hiện được vật thể nguy hiểm
    \item Xem lại các ảnh đã được capture trong Gallery
\end{itemize}

\section{Kết luận}

Dự án đã hoàn thành các mục tiêu đề ra:
\begin{itemize}
    \item Fine-tune YOLOv11 để nhận diện dao và kéo
    \item Xây dựng web application với real-time detection
    \item Tự động capture khi phát hiện vật thể nguy hiểm
\end{itemize}

\end{document}
